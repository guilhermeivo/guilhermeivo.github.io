\documentclass{article}

\usepackage[brazil]{babel}
\usepackage[T1]{fontenc}
\usepackage[utf8]{inputenc}

\usepackage{graphicx}
\usepackage{float}

\usepackage[section, maketitle]{m_style}

\usepackage{listings}
\renewcommand{\lstlistingname}{Código}
\lstdefinestyle{m_lststyle}{
    % Colors
    backgroundcolor = \color{white},
    rulesepcolor    = \color{darkgray}, 
    identifierstyle = \color{black},
    commentstyle    = \color{gray},
    % Caracteres
    basicstyle      = \ttfamily\footnotesize,
    extendedchars   = true,
    % Frame Style
    numbers         = left,
    tabsize         = 2,
    numberstyle     = \scriptsize,
    frame           = shadowbox,
    breaklines      = true,
    breakautoindent = true,
    xleftmargin     = 25pt,
    xrightmargin    = 25pt,
    aboveskip       = 10pt,
    belowskip       = 10pt,
    belowcaptionskip = 5pt,
    captionpos      = t,
    morekeywords    = {matrix4, vector3, real_t, opmat4, glm, mat4, lookAt, affineInverse, ortho, invert}}
\usepackage[bottom]{footmisc} % force footnote in bottom

\title{Comparação Javascript \(\boldsymbol{\times}\) WebAssembly}

\hypersetup{
    colorlinks = true,
    allcolors = black
}

\begin{document}

\maketitle\pagenumbering{arabic}

\section*{Introdução}

Este documento é usado como forma de demonstrar as motivações para o uso do \textit{WebAssembly}\cite{wasi} em substituição do código nativo em \textit{Javascript} nos métodos para cálculos de matrizes (\verb|Matrix4.js| ou \verb|matrix4.h|) e vetores (\verb|Vector3.js| ou \verb|vector3.h|) no projeto atual. A nova implementação - em \textit{WebAssembly} - é encontrada a partir da \textit{commit} \href{https://github.com/guilhermeivo/guilhermeivo.github.io/commit/2b537169b3fdd6d0d027cc9d810aac9471eb477e}{2b53716} e, como consequência, a antiga implementação - em \textit{Javascript} - é encontrada anteriormente à essa \textit{commit}. Serão compilados para \textit{WebAssembly} códigos escritos em \textit{C}/\textit{C++} baseados nos recursos utilizados no código fonte da implementação \href{https://github.com/WebAssembly/wasi-libc}{wasi-libc}, como, de importar e exportar as funções usando \verb|__attribute__|.

Os testes são, principalmente, de comparações entre o tempo de execução e tamanho do arquivo binário do \textit{Javascript} e \textit{WebAssembly}, com diferentes implementações, e podem ser enumerados, em um hierarquia temporal de realização, da seguinte maneira:

{\renewcommand{\labelenumi}{(\alph{enumi})}
\begin{enumerate}
    \item Formas de inicializar um vetor usando \textit{C++};
    \item Cálculo da matriz nomeada \textit{projectionView}\footnote{multiplicação entre a matriz de projeção (\textit{projection}) e matriz de visualização (\textit{view}) é usada para definir a posição do vértice do objeto};
    \item Cálculo da mesma matriz, porém, dessa vez, usando o pacote OpenGL Mathematics (GLM)\cite{glm}.
\end{enumerate}}

\section*{Inicialização Vetor}

Observando a inicialização de um vetor com valores padrões, por exemplo, 0, em um vetor com 16 posições, não é possível observar uma expressiva diferença quanto a velocidade de preenchimento desses valores, mesmo quando observado os métodos de otimização dos arquivos binários, como observado na \autoref{fig:eff-init}. Portanto, não terá um caráter rigoroso quanto a inicialização utilizada, seguindo a forma de otimização -O3 nos arquivos binários para gerar códigos mais rápidos (não me preocupando, nesse caso e nesse momento, com o tamanho do arquivo, pois foram criadas poucas instruções nos arquivos finais utilizados resultando em um arquivo de, aproximadamente, 60 kB).
% NOTE: tamanho do arquivo inserido de forma manual

\begin{figure}[H]
    \centering
    \input{eff_init/generated/caption.tex}
    \vspace{5pt}
    \includegraphics[width = 0.9\linewidth]{eff_init/generated/o__boxplot.pdf}
    \vspace{5pt}
    {\par\small Fonte: Elaborado pelo próprio autor}
    \input{eff_init/generated/description.tex}
    \label{fig:eff-init}
\end{figure}

\section*{Cálculo da Matriz projectionView no NodeJS}

A multiplicação entre uma matriz de projeção e uma matriz de visualização (\autoref{lst:projectionView}) é usado no \textit{Shaders OpenGL} para definir a posição do vértice do objeto, sendo, então, um código necessário de ser executado a cada \textit{frame}, portanto não pode apresentar grandes dificuldades na sua chamada. E então, a grande motivação para substituir o uso dos códigos \textit{Javascript} em cálculos matriciais e vetoriais por códigos em \textit{WebAssembly}.

\begin{lstlisting}[language=c,style=m_lststyle,label=lst:projectionView,caption=Cálculo da Matriz projectionView]
// perspective or projection matrix
matrix4<real_t> projectionMatrix.orthographic(
    -orthographicUnits * aspect, // left
    orthographicUnits * aspect,  // right
    -orthographicUnits,          // bottom
    orthographicUnits,           // top
    zNear,
    zFar)

matrix4<real_t> cameraMatrix = opmat4::lookAt<real_t>(position, target, yAxis);

// faz uma matriz de visualizacao a partir da matriz da camera
matrix4<real_t> viewMatrix = opmat4::invert<real_t>(cameraMatrix);

// mova o espaco de projecao para visualizar o espaco (o espaco na frente da camera)
// perspective ou projection matrix * view matrix
matrix4<real_t> projectionViewMatrix = projectionMatrix * viewMatrix;
\end{lstlisting}
% NOTE: codigo inserido de forma manual

Segundo a \autoref{fig:time-run} o uso do \textit{Javascript} é \input{time_run/generated/comparison.tex} vezes mais lento que o uso do \textit{WebAssembly}\footnote{Esses dados foram gerados no \textit{NodeJS}, possívelmente, não apresentará os mesmos valores no navegador}.

\begin{figure}[H]
    \centering
    \input{time_run/generated/caption.tex}
    \vspace{5pt}
    \includegraphics[width = 0.9\linewidth]{time_run/generated/o__boxplot.pdf}
    \vspace{5pt}
    {\par\small Fonte: Elaborado pelo próprio autor}
    \input{time_run/generated/description.tex}
    \label{fig:time-run}
\end{figure}

\section*{Cálculo da Matriz projectionView no C++}

Apenas para finalidades de verificação da velocidade de execução dos códigos desenvolvidos (\verb|Matrix4.h| e \verb|Vector3.h|) será comparado com a biblioteca de cálculos matemáticos \textit{GLM}, utilizada em diversos programas de produção. Observado em alguns testes uma melhor velocidade usando \textit{GLM} e em outros usando o código desenvolvido, como visto na \autoref{fig:glm}, portanto uma inexpressiva diferença podendo manter assim o códigos desenvolvidos.

\begin{lstlisting}[language=c,style=m_lststyle,label=lst:projectionViewGLM,caption=Cálculo da Matriz projectionView usando GLM]
glm::mat4 projectionMatrix = glm::ortho(
-orthographicUnits * aspect, // left
orthographicUnits * aspect,  // right
-orthographicUnits,          // bottom
orthographicUnits,           // top
zNear,
zFar);

glm::mat4 cameraMatrix = glm::lookAt(position, target, yAxis);

glm::mat4 viewMatrix = glm::affineInverse(cameraMatrix);

glm::mat4 projectionViewMatrix = projectionMatrix * viewMatrix;
\end{lstlisting}
% NOTE: codigo inserido de forma manual

\begin{figure}[H]
    \centering
    \input{glm/generated/caption.tex}
    \vspace{5pt}
    \includegraphics[width = 0.9\linewidth]{glm/generated/o__boxplot.pdf}
    \vspace{5pt}
    {\par\small Fonte: Elaborado pelo próprio autor}
    \input{glm/generated/description.tex}
    \label{fig:glm}
\end{figure}

\newpage\pagestyle{empty}\pagenumbering{gobble}
\bibliographystyle{plain}
\bibliography{references}

\end{document}